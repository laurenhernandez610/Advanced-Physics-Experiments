\documentclass[14pt]{beamer}
\usepackage{graphicx}
\usepackage{booktabs}
\usepackage{xcolor}
\usepackage[absolute,overlay]{textpos}
\usepackage{wrapfig}
\usetheme{CambridgeUS}
\newcommand{\putat}[3]{\begin{picture}(0,0)(0,0)\put(#1,#2){#3}\end{picture}}
\logo{\includegraphics[width=20mm]{u-houston-logo.png}}
\title{PHOTODIODES}
\author{LAUREN HERNANDEZ}

\medskip

\begin{document}

%      Title Page
% -------------------------------


\begin{frame}
\titlepage
\end{frame}


%      Table of Contents 
% ------------------------------
	
\begin{frame}{OVERVIEW}
\tableofcontents

\begin{flushleft}
\begin{itemize}
	\item What is a photodiode?
	\vspace{5mm}
	\item History
	\vspace{5mm}
	\item Various uses
	\vspace{5mm}
	\item How does it work?
	\vspace{5mm}
	\item How do you operate it?
	\vspace{5mm}
	\item Conclusion
\end{itemize}
\end{flushleft}
\end{frame}


%      What is a Photodiode
% ------------------------------
\begin{frame}
	
	\frametitle{WHAT IS A PHOTODIODE}
	\begin{flushleft}
	\begin{list}{--}{}
		\item semiconductor
		\item light $\rightarrow$ electricity
		\item detect light intensity
	\end{list}
	\end{flushleft}	
	\vspace{5mm}
	\begin{figure}[H]
	\begin{flushleft}
	\includegraphics[width=70mm]{Introduction-to-Photodiode.jpeg}
	\end{flushleft}
	\end{figure}

\begin{textblock*}{6cm}(7cm,2cm) % {block width} (coords)
	\includegraphics[width=5.5cm]{light1.jpeg}
\end{textblock*}

\begin{textblock*}{3cm}(7.5cm,6.5cm) % {block width} (coords)
	\includegraphics[width=3cm]{diode.jpeg}
\end{textblock*}

\begin{textblock*}{3cm}(9cm,4.5cm) % {block width} (coords)
	\includegraphics[width=3cm]{electricity.jpeg}
\end{textblock*}
	
\end{frame}



% Brief History of the Photodiode
% ----------------------------------
\begin{frame}{BRIEF HISTORY}
	
		\begin{flushleft}
		\begin{list}{--}{}
			\item Dr. John N. Shive
			\item Bell Labs, 1948
			\item Nobel for transistor
		\end{list}
	\end{flushleft}

	\begin{figure}[H]
	\begin{flushleft}
		\includegraphics[width=59mm]{bell labs.jpeg}
	\end{flushleft}
\end{figure}

\begin{textblock*}{3cm}(6.5cm,1.75cm) % {block width} (coords)
	\includegraphics[width=3.cm]{John.jpeg}
\end{textblock*}

\begin{textblock*}{3cm}(9.75cm,1.75cm) % {block width} (coords)
	\includegraphics[width=3cm]{bell.jpeg}
\end{textblock*}

\begin{textblock*}{3cm}(6.75cm,5.75cm) % {block width} (coords)
	\includegraphics[width=3.5cm]{transistor.jpeg}
\end{textblock*}	
	



\end{frame}

%   What are the uses of a Photodiode
% ------------------------------
\begin{frame}{VARIOUS USES}
	
		\begin{figure}[H]
		\begin{flushleft}
			\includegraphics[width=45mm]{lights.png}
		\end{flushleft}
	\end{figure}

	\begin{textblock*}{3cm}(5.45cm,1.75cm) % {block width} (coords)
	\includegraphics[width=2.75cm]{smoke.jpeg}
\end{textblock*}

	\begin{textblock*}{3cm}(8.5cm,1.75cm) % {block width} (coords)
	\includegraphics[width=4cm]{xray.jpeg}
\end{textblock*}
	
		\begin{figure}[H]
		\begin{flushleft}
			\includegraphics[width=77mm]{solar cell.jpeg}
		\end{flushleft}
	\end{figure}
	
\end{frame}

%   How does a Photodiode actually work
% ------------------------------
\begin{frame}{HOW DOES IT WORK?}
	
		\begin{textblock*}{4cm}(.1cm,1.75cm) % {block width} (coords)
		\includegraphics[width=6cm]{photoelectric.png}
	\end{textblock*}

		\begin{textblock*}{4cm}(.01cm,5cm) % {block width} (coords)
	\includegraphics[width=5cm]{pn diagram.png}
\end{textblock*}
	
			\begin{textblock*}{4cm}(5.5cm,3cm) % {block width} (coords)
		\includegraphics[width=6.75cm]{image.png}
	\end{textblock*}
\end{frame}

%   How does a Photodiode actually work
% ------------------------------
\begin{frame}{HOW DOES IT WORK?}
	
			\begin{flushleft}
		\begin{list}{--}{}
			\item PN 
			\item PIN
			\item Avalanche
			\item Schottky 
		\end{list}
	\end{flushleft}	
	\vspace{5mm}
	
	\begin{figure}[H]
		\begin{flushleft}
			\includegraphics[width=77mm]{explanation-of-photodiodes.jpeg}
		\end{flushleft}
	\end{figure}
	
	\begin{textblock*}{4cm}(6.5cm,1.5cm) % {block width} (coords)
		\includegraphics[width=6cm]{pn.jpeg}
	\end{textblock*}
	
\end{frame}


%   How does it work?
% ------------------------------
\begin{frame}{HOW DOES IT WORK?}
		
		\begin{flushleft}
		\quad \quad	\underline{PERFORMANCE PARAMETERS}
		\vspace{3mm}
		\begin{list}{-}{}
			\item Responsivity = $\frac{\text{generated photocurrent}}{\text{light intensity}}$
			\vspace{1mm}
			\item Quantum efficiency = $\frac{\text{electron hole pairs}}{\text{incident photons}}$
			\vspace{1mm}
			\item Response time 
 
		\end{list}
	\end{flushleft}	
	\vspace{5mm}
	
\end{frame}



%   How do you operate, put it into action?
% ------------------------------
\begin{frame}{HOW DO YOU OPERATE IT?}
	
		\begin{flushleft}
		\begin{list}{--}{}
			\item Light 
			\item Circuit 
			\item Photoconductive mode
			\item Photovoltaic mode 
		\end{list}
	\end{flushleft}	

	\begin{figure}[H]
	\begin{flushleft}
		\includegraphics[width=85mm]{images.png}
	\end{flushleft}
\end{figure}
	
\end{frame}


%   Slide
% ------------------------------
\begin{frame}{CONCLUDING THOUGHTS}
	
			\begin{flushleft}
		\begin{list}{--}{}
			\item Turn photons into current
			\item Invented alongside transistors
			\item Wide variety of uses
			\item Can be adjusted for effeciency
			\item Can produce voltage or current
		\end{list}
	\end{flushleft}	
	
		\begin{figure}[H]
		\begin{flushleft}
			\includegraphics[width=85mm]{conclusion.jpeg}
		\end{flushleft}
	\end{figure}
	
\end{frame}



%   Slide
% ------------------------------
\begin{frame}{REFERENCES}
	
	\begin{flushleft}
		\begin{list}{--}{}
			
	\item \scriptsize{Photodiode operation, types and applications. ElectricalMag. (2020, August 22). Retrieved October 19, 2021, from \color{blue}{https://electricalmag.com/photodiode/. }}
	
	\item \scriptsize{TUFAN, B. H. (2020, May 10). Explanation of photodiodes. Thecodeprogram. Retrieved October 19, 2021, \color{blue}{from https://thecodeprogram.com/explanation-of-photodiodes. }}
	
	\item \scriptsize{P-n junction. The P-N Junction. (n.d.). Retrieved October 19, 2021, from \color{blue}{http://hyperphysics.phy-astr.gsu.edu/hbase/Solids/pnjun.html. }}
	
	\item \scriptsize{Department of Chemistry | UCI Department of Chemistry. (n.d.). Retrieved October 19, 2021, from \color{blue}{https://www.chem.uci.edu/~unicorn/243/handouts/photodiode.pdf. }}
	
	\item \scriptsize{YouTube. (2020, December 11). Photodiodes - (Working \&amp; Why it's reverse biased) | semiconductors | physics | khan academy. YouTube. Retrieved October 19, 2021, from \color{blue}{https://www.youtube.com/watch?v=KgKcbW77txY. } }
	
	\vspace{10mm}
	\end{list}
	\end{flushleft}
\end{frame}





\end{document}