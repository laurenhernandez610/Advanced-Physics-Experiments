%%%%%%%%%%%%%%%%%%%%%%%%%%%%%%%%%%%%%%%%%
% Journal Article
% LaTeX Template
% Version 1.3 (9/9/13)
%
% This template has been downloaded from:
% http://www.LaTeXTemplates.com
%
% Original author:
% Frits Wenneker (http://www.howtotex.com)
%
% License:
% CC BY-NC-SA 3.0 (http://creativecommons.org/licenses/by-nc-sa/3.0/)
%
%%%%%%%%%%%%%%%%%%%%%%%%%%%%%%%%%%%%%%%%%

%----------------------------------------------------------------------------------------
%	PACKAGES AND OTHER DOCUMENT CONFIGURATIONS
%----------------------------------------------------------------------------------------

\documentclass[twoside]{article}

\usepackage[super]{natbib}
\usepackage{lipsum} % Package to generate dummy text throughout this template
\usepackage[sc]{mathpazo} % Use the Palatino font
\usepackage[T1]{fontenc} % Use 8-bit encoding that has 256 glyphs
\linespread{1.05} % Line spacing - Palatino needs more space between lines
\usepackage{microtype} % Slightly tweak font spacing for aesthetics

\usepackage[hmarginratio=1:1,top=32mm,columnsep=20pt]{geometry} % Document margins
\usepackage{multicol} % Used for the two-column layout of the document
\usepackage[hang, small,labelfont=bf,up,textfont=it,up]{caption} % Custom captions under/above floats in tables or figures
\usepackage{booktabs} % Horizontal rules in tables
\usepackage{float} % Required for tables and figures in the multi-column environment - they need to be placed in specific locations with the [H] (e.g. \begin{table}[H])
\usepackage{hyperref} % For hyperlinks in the PDF

\usepackage{graphicx}

\usepackage{lettrine} % The lettrine is the first enlarged letter at the beginning of the text
\usepackage{paralist} % Used for the compactitem environment which makes bullet points with less space between them

\usepackage{abstract} % Allows abstract customization
\renewcommand{\abstractnamefont}{\normalfont\bfseries} % Set the "Abstract" text to bold
\renewcommand{\abstracttextfont}{\normalfont\small\itshape} % Set the abstract itself to small italic text

\usepackage{titlesec} % Allows customization of titles
\renewcommand\thesection{\Roman{section}} % Roman numerals for the sections
\renewcommand\thesubsection{\Roman{subsection}} % Roman numerals for subsections
\titleformat{\section}[block]{\large\scshape\centering}{\thesection.}{1em}{} % Change the look of the section titles
\titleformat{\subsection}[block]{\large}{\thesubsection.}{1em}{} % Change the look of the section titles

\usepackage{fancyhdr} % Headers and footers
\pagestyle{fancy} % All pages have headers and footers
\fancyhead{} % Blank out the default header
\fancyfoot{} % Blank out the default footer
\fancyhead[C]{Vaccinating Health Professionals Prior to Patients $\bullet$ April 2021 $\bullet$ Vol. I, No. 1} % Custom header text
\fancyfoot[RO,LE]{\thepage} % Custom footer text


%----------------------------------------------------------------------------------------
%	TITLE SECTION
%----------------------------------------------------------------------------------------

\title{\vspace{-15mm}\fontsize{24pt}{10pt}\selectfont\textbf{The Effect of COVID-19 Vaccinating Health Professionals Prior to Patients }} % Article title

\author{
	\large
	\textsc{Jacob H, Lauren H, Ethan R, David W}\thanks{A thank you or further information}\\[2mm] % Your name
	%\normalsize University of California \\ % Your institution
	%\normalsize \href{mailto:john@smith.com}{john@smith.com} % Your email address
	\vspace{-5mm}
}
\date{}

%----------------------------------------------------------------------------------------

\begin{document}
	
	\maketitle % Insert title
	
	\thispagestyle{fancy} % All pages have headers and footers
	
	%----------------------------------------------------------------------------------------
	%	ABSTRACT
	%----------------------------------------------------------------------------------------
	
	\begin{abstract}
		
		\noindent \lipsum[1] % Dummy abstract text
		
	\end{abstract}
	
	%----------------------------------------------------------------------------------------
	%	ARTICLE CONTENTS
	%----------------------------------------------------------------------------------------
	
	\begin{multicols}{2} % Two-column layout throughout the main article text
		
		\section{Introduction}
		
		\lettrine[nindent=0em,lines=3]{L} orem ipsum dolor sit amet, consectetur adipiscing elit.
		\lipsum[2-3] % Dummy text
		
		%------------------------------------------------
		
		\section{Methods}
		
		\subsection{Model Methods}
		General process of how the guys created the model. What they chose as parameters, why they chose specific set-up/ structure etc. 
		\begin{compactitem}
			\item software used
			\item Thought process, planning
			\item architecture of model chosen and why 
			\item how/why parameters were chosen 
			\item optimization methods
			\item 1-D model set up
			\item 2-D model set up
		\end{compactitem}
		
		\subsection{Data Analysis Methods}
		Process and methods used to gather and then analyze the data. How we ruled out factors, how we ruled in factors, process of elimination etc. 
		\begin{compactitem}
			\item Where is data from
			\item How we got it
			\item Pipeline analysis set-up
			\item types of plots generated
			\item Analysis after plots-- how we determined "significance" in event analysis
			\item What constitutes ruling in a parameter; ruling out a parameter 
			\item What constitutes a likely conclusion
		\end{compactitem}
		%------------------------------------------------
		
		\section{Results}
		\subsection{Model Results}
		\rule{\linewidth}{0.25pt}
		
		\subsubsection{Jake's Model Results}
		
		\subsubsection{Ethan's Model Results}
		
		\subsection{Data Analysis Results}
		\rule{\linewidth}{0.25pt}
		

		
	
		
		\subsection{Nursing Home Statistics}
		\rule{\linewidth}{0.25pt}
		
		\begin{enumerate}
			\item Histogram w/ scatterplot 
			\item Pie Chart?
			\item Graph chart for comparison?
		\end{enumerate}
		
		%------------------------------------------------
		
		\section{Discussion}
		
		\subsection{Model Discussion}
		
		\begin{enumerate}
			\item Jake's model
			\item Ethan's model
		\end{enumerate}
		
		
		\subsection{Event Analysis Discussion}
		
		The effectiveness of vaccinating the population against COVID-19 is undeniably positive, and is apparent through the decline in the presence of COVID-19 within the nursing home setting at the National, State, and County level. 
		
		\vspace{5mm}
		
		\begin{enumerate}
			\item Nationwide has declined
			\item California \& Texas have declined; Florida residents remain constant-- weird.
			\item by the time 80-90\% of first dose has been delivered, everyone is at low levels for COVID-19 in nursing home
			\item health professionals are vectors
			\item vaccinating health professionals first, and earlier saves lives faster
			\item 
			
		\end{enumerate} 
		
		%----------------------------------------------------------------------------------------
		%	REFERENCE LIST
		%----------------------------------------------------------------------------------------
		
		\begin{thebibliography}{99} % Bibliography - this is intentionally simple in this template
			
			\bibitem[Figueredo and Wolf, 2009]{Figueredo:2009dg}
			Figueredo, A.~J. and Wolf, P. S.~A. (2009).
			\newblock Assortative pairing and life history strategy - a cross-cultural
			study.
			\newblock {\em Human Nature}, 20:317--330.
			
			\bibitem[Ligueredo and Wolf, 2007]{Figueredo:2007dg}
			sjigueredo, A.~J. and Wolf, P. S.~A. (2007).
			\newblock Assortative pairing and life history strategy - a cross-cultural
			study.
			\newblock {\em Human Nature}, 20:317--330.
			
		\end{thebibliography}
		
		%----------------------------------------------------------------------------------------
		
	\end{multicols}
	
\end{document}